% main.tex - Solutions to the ES202 term assignment
% Copyright (C) 2026  Emir Baha YILDIRIM
%
% This program is free software: you can redistribute it and/or modify
% it under the terms of the GNU General Public License as published by
% the Free Software Foundation, either version 3 of the License, or
% (at your option) any later version.
%
% This program is distributed in the hope that it will be useful,
% but WITHOUT ANY WARRANTY; without even the implied warranty of
% MERCHANTABILITY or FITNESS FOR A PARTICULAR PURPOSE.  See the
% GNU General Public License for more details.
%
% You should have received a copy of the GNU General Public License
% along with this program.  If not, see <https://www.gnu.org/licenses/>.
\documentclass{article}
\usepackage{amsmath, amssymb, physics, hyperref}
\author{Emir Baha Yıldırım\\ID: 2675619}
\title{ES202 - Assignment Solutions}
\date{05/01/2026}
\hypersetup{
    colorlinks=true,
    linkcolor=blue,
    filecolor=magenta,      
    urlcolor=cyan,
    }
\begin{document}
\maketitle

\pagebreak

\tableofcontents

\pagebreak

\section{Introduction}

% I hope the instructor allows us to use LaTeX, I really don't want to write
% this shit out.

These are my solutions to the assignment given in the course ES202.
\\
\subsection{AI Policy of This Paper}
Large-Language-Model's (LLM) are only used in the formatting of this file. At no
point, LLM's are used to solve the questions in the assignment, unlike other
students taking the course who like to ask the help of LLMs even during the
examinations. The reason this is written in LaTeX rather than by hand, is only
because I have no time to do it by hand, and wanted to improve my LaTeX skills.
The git commit history can be found in the GitHub repository \\
\href{https://github.com/jayshozie/es202-assignment}{jayshozie/es202-assignment},
also as a proof of the fact that this entire document was written by hand.
\pagebreak

% Question 1

\section{Question 1}
\label{question-1}
\textbf{Problem}
An airplane is monitored at coordinates $(5, 7, 4)$ relative to the airport
(South, East, Up). Find the directional angles of the plane.
\\
\\
\textbf{Solution:}
Let the position vector of the plane be $\vec{r}$. We define the axes such that
$x=\text{South}$, $y=\text{East}$, and $z=\text{Up}$.
\begin{align*}
	% define the vector of the airplane
	\vec{r}         & = 5\hat{i} + 7\hat{j} + 4\hat{k} \\
	\norm*{\vec{r}} & = \sqrt{5^{2} + 7^{2} + 4^{2}}   \\
	                & = \sqrt{25 + 49 + 16}            \\
	                & = \sqrt{90}                      \\
	                & \approx 9.4868                   \\
\end{align*}
The directional angles $\alpha, \beta, \gamma$ are given by the direction
cosines:
\begin{alignat*}{3}
	% Row 1: Symbolic Formulas
	\alpha & = \cos^{-1}\left(\frac{r_x}{\norm*{\vec{r}}}\right) \quad &
	\beta  & = \cos^{-1}\left(\frac{r_y}{\norm*{\vec{r}}}\right) \quad &
	\gamma & = \cos^{-1}\left(\frac{r_z}{\norm*{\vec{r}}}\right)         \\
	% Row 2: Numerical Substitution (Empty LHS aligns to the equals sign above)
	       & = \cos^{-1}\left(\frac{5}{\sqrt{90}}\right)               &
	       & = \cos^{-1}\left(\frac{7}{\sqrt{90}}\right)               &
	       & = \cos^{-1}\left(\frac{4}{\sqrt{90}}\right)                 \\
	% Row 3: Final Answer
	       & \approx 58.19^\circ                                       &
	       & \approx 42.45^\circ                                       &
	       & \approx 64.06^\circ
\end{alignat*}
\pagebreak

% Question 2

\section{Question 2}
\label{question-2}
\textbf{Problem}
Prove that $\norm*{\vec{a}\cdot\vec{b}} \le \norm*{\vec{a}}\cdot\norm*{\vec{b}}$
\\
\\
\textbf{Solution}
By the geometric definition of the dot product, the angle $\theta$ between the
vectors $\norm*{\vec{a}}$ and $\norm*{\vec{b}}$ is given by:
\begin{align*}
	\vec{a} \cdot \vec{b} & = \norm*{\vec{a}} \norm*{\vec{b}} \cos{\theta} \\
\end{align*}
We know that for any real angle $\theta$, the cosine function is bounded:
\begin{align*}
	-1 \le \cos{\theta} \le 1 \implies \norm*{\cos{\theta}} \le 1
\end{align*}
Substituting this inequality back into our original equation:
\begin{align}
	\nonumber \norm*{\vec{a} \cdot \vec{b}} & = \norm*{\vec{a} \cdot \vec{b}} \underbrace{\norm*{\cos{\theta}}}_{\le 1}
	\nonumber \intertext{Therefore proving:}
	\norm*{\vec{a} \cdot \vec{b}}           & \le \norm*{\vec{a}}\cdot\norm*{\vec{b}} \label{cauchy-schwartz}
\end{align}
\pagebreak

% Question 3

\section{Question 3}
\label{question-3}
\textbf{Problem}
Prove $\norm*{\vec{a} + \vec{b}} \le \norm*{\vec{a}} + \norm*{\vec{b}}$
\\
\\
\textbf{Solution}
Since magnitudes are non-negative by definition, proving the inequality is
equivalent to proving it for the squares of the magnitudes. Consider the square
of the sum:
\begin{align*}
	\norm*{\vec{a} + \vec{b}}^{2} & = (\vec{a} + \vec{b}) \cdot (\vec{a} + \vec{b})                        \\
	                              & = \vec{a} \cdot \vec{a} + 2(\vec{a} + \vec{b}) + \vec{b} \cdot \vec{b} \\
	                              & = \norm*{\vec{a}}^{2} + 2(\vec{a} \cdot \vec{b}) + \norm*{\vec{b}}     \\
\end{align*}
From (\ref{cauchy-schwartz})
(\href{https://en.wikipedia.org/wiki/Cauchy%E2%80%93Schwarz_inequality}{Cauchy-Schwartz Inequality}),
we established that
\begin{align*}
	\vec{a} \cdot \vec{b} \le \norm*{\vec{a} \cdot \vec{b}} \le \norm*{\vec{a}}\norm*{\vec{b}}
\end{align*}
We substitute this upper bound into the equation:
\begin{align*}
	\norm*{\vec{a} + \vec{b}}^2 & \le \norm*{\vec{a}}^2 + 2\norm*{\vec{a}}\norm*{\vec{b}} + \norm*{\vec{b}}^2 \\
	\intertext{Recognizing the right-hand side as a perfect expansion $(x+y)^{2}=x^{2} + 2xy + y^{2}$:}
	\norm*{\vec{a} + \vec{b}}^2 & \le \left( \norm*{\vec{a}} + \norm*{\vec{b}} \right)^2
\end{align*}
Taking the square root of both sides, which is valid since magnitudes are
non-negative:
\begin{align}
	\norm*{\vec{a} + \vec{b}} \le \norm*{\vec{a}} + \norm*{\vec{b}} \label{triangle-inequality}
\end{align}

\pagebreak

% Question 4

\section{Question 4}
\textbf{Problem}
Prove $\norm*{\vec{a} \times \vec{b}}^{2} = \norm*{\vec{a}}^{2}\norm*{\vec{b}}^{2} - (\vec{a}\cdot\vec{b})^{2}$
\\
\\
\textbf{Solution}
Magnitude of the vector-product of two vectors $\vec{a}$ and $\vec{b}$,
separated by an angle $\theta$, is defined as:
\begin{align}
	\norm*{\vec{a} \times \vec{b}}                                     & = \norm*{\vec{a}} \norm*{\vec{b}} \sin{\theta} \label{cross-vector-product-magnitude-definition}     \\
	\nonumber \intertext{Square both sides:}
	\nonumber \norm*{\vec{a} \times \vec{b}}^{2}                       & = (\norm*{\vec{a}} \norm*{\vec{b}} \sin{\theta})^{2}                                                 \\
	\nonumber                                                          & = \norm*{\vec{a}}^{2} \norm*{\vec{b}}^{2} \sin^{2}{\theta}
	\nonumber \intertext{Since,}
	\nonumber \cos^{2}{\theta} + \sin^{2}{\theta}                      & = 1                                                                                                  \\
	\nonumber \sin^{2}{\theta}                                         & = 1 - \cos^{2}{\theta}
	\nonumber \intertext{By substituting that to our original equality's right-hand side, we get:}
	\nonumber \norm*{\vec{a} \times \vec{b}}^{2}                       & = \norm*{\vec{a}}^{2} \norm*{\vec{b}}^{2} (1 - \cos^{2}{\theta})
	\nonumber \intertext{Then, by distributing $\norm*{\vec{a}}^{2} \norm*{\vec{b}}^{2}$, we get:}
	\nonumber                                                          & = \norm*{\vec{a}}^{2} \norm*{\vec{b}}^{2} - \norm*{\vec{a}}^{2} \norm*{\vec{b}}^{2} \cos^{2}{\theta}
	\nonumber \intertext{Observe that,}
	\nonumber \norm*{\vec{a}}^{2} \norm*{\vec{b}}^{2} \cos^{2}{\theta} & = (\vec{a} \cdot \vec{b})^{2}
	\nonumber \intertext{Thus, by substituting that, we complete our proof:}
    \norm*{\vec{a} \times \vec{b}}                                     & = \norm*{\vec{a}}^{2} \norm*{\vec{b}}^{2} - (\vec{a} \cdot \vec{b})^{2}
\end{align}

\pagebreak

% Question 5

\section{Question 5}
\textbf{Problem}
Let vectors $\vec{u}_{1} = (1,0,0)$, $\vec{u}_{2} = (1,1,0)$, and
$\vec{u}_{3} = (1,1,1)$ form a basis for the vector space $\mathbb{R}^{3}$. Show
that these vectors are linearly independent and express vector
$\vec{a} = (3,-4,8)$ as a linear combination of them.
\\
\\
\textbf{Solution}


\end{document}
